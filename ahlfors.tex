\documentclass[a4paper,12pt,openany]{book}
\usepackage{glimpse}

\title{Annotations : Complex Analysis\\2nd Edition, Lars Valerian Ahlfors}
\author{Jacob Antony\\jacobantony987@gmail.com}

\begin{document}
\maketitle

\chapter{The Algebra of Complex Numbers}
\textbf{\phantom{}}\marginpar{\S1.1 pp. 1}
\textbf{Addition and multiplication do not lead out from the system of complex numbers.}\\
\begin{story}
	If we make a larger number system containing real numbers, then we want to add/multiply real numbers in the new system the same way as before.
\end{story}

\textbf{\phantom{}}\marginpar{\S1.1 pp. 2}
\textbf{Once the existence of quotient has been proved, its value can be found in a simpler way.}\\
\begin{story}
	Suppose $x,y \in \mathbb{C}$.
	$x/y$ exists only if there exists a unique number $z \in \mathbb{C}$ such that $x = yz$.
	For example, $3/5$ exists only if there exists $x$ such that $5x = 3$.
	And $3/5$ is undefined in $\mathbb{Z}$ since there is no integer $x$ such that $5x = 3$.
	At the same time, $3/5$ is undefined since $0.6$ is not an integer is a very bad logic.
\end{story}

\textbf{\phantom{}}\marginpar{\S1.2 pp. 3}
\textbf{\dots But these values cannot be combined arbitrarily, for the second equation (4) is not a consequence of (5).}\\
\begin{story}
	Suppose $\beta \ne 0$, then we have four possible soultions to (5).
	But, all of them does not satisfy $2xy = \beta$.
	It turns out that $x$ and $y$ has same sign if $\beta > 0$.
	And, $x$ and $y$ has different signs if $\beta < 0$.
\end{story}

\textbf{\phantom{}}\marginpar{\S1.2 pp. 4}
\textbf{\dots it is not possible to distinguish between the positive and negative square root of a complex number.}\\
\begin{story}
	Every non-zero complex number have two complex numbers as square roots.
	But, complex numbers can not have an order relation.
	Therefore, we cannot name those roots positive or negative.
\end{story}

\textbf{\phantom{}}\marginpar{\S1.3 pp. 4}
\textbf{So far our approach to complex numbers has been completely uncritical.}\\
\begin{story}
	Before defining the operations on $\mathbb{C}$, we should have proved the existence of such a number system ( which is algebraically unique ).
	Author delayed that proof, probably because he doesn't want it to be a distraction.
\end{story}


%\bibliographystyle{ieeetr}
\bibliographystyle{apalike}
\bibliography{reference}
\end{document}
